\documentclass{article}
\usepackage{packages}

\title{Коллоквиум по Математическому анализу-2, семестр 2}
\author{Виноградова Дарья, Залялов Александр, Миронов Алексей, Стрельцов Артём, Т}
\date{}

\begin{document}

	\maketitle

	\tableofcontents

	\clearpage


	\setcounter{section}{13}
\section{Дайте определение $k$-мерного подмногообразия в $\mathbb{R}^n$ и сопутствующее определение гладких координат. Приведите пример параметрической кривой, которая параметрически задана дифференцируемыми функциями, но не является гладким 1-мерным многообразием в какой-нибудь точке}
Подмножество $M\subseteq \mathbb{R}^n$ называется гладким $k$-мерным (под)многообразием в $\mathbb{R}^n$, если $\forall x \in M$ существует окрестность $U$, $x\in U$, такая что на $M \cap U$ можно задать гладкие координаты.
Гладкие координаты --- отображение $\Phi: V \xrightarrow{} M$, где $V \subseteq \mathbb{R}^k$
\begin{gather*}
\begin{cases}
    x_1 = \phi_1(t_1, \dotsc, t_k)\\
    \vdots\\
    x_n = \phi_n(t_1, \dotsc, t_k)
\end{cases}
\end{gather*}
где $(x_1, \dotsc, x_n)$ --- координаты в $\mathbb{R}^n$, $(t_1, \dotsc, t_k)$ --- координаты в $\mathbb{R}^k$,
при этом $(t_1, \dotsc, t_k) \in V$ тогда и только тогда, когда $(x_1, \dotsc, x_n)\in M$. При этом $\phi_1,\dotsc,\phi_n$ дифференцируемы по каждой переменной и матрица частных производных невырождена.
\begin{gather*}
\begin{pmatrix}
\frac{\partial \phi_1}{\partial t_1} & \frac{\partial \phi_2}{\partial t_1} & \cdots & \frac{\partial \phi_n}{\partial t_1} \\
\vdots & \vdots & \ddots & \vdots \\
\frac{\partial \phi_1}{\partial t_k} & \frac{\partial \phi_2}{\partial t_k} & \cdots & \frac{\partial \phi_n}{\partial t_k}
\end{pmatrix}
\end{gather*}
Ранг этой матрицы должен быть $k$ в любой точке $t \in V$ (то есть все строчки должны быть линейно независимы.\\\\
Пример:
\begin{gather*}
\begin{cases}
    x=t^2\\
    y=t^3
\end{cases}
\end{gather*}
Обе функции дифференцируемые, но в точке $t=0$ обе производные обращаются в ноль. Поэтому кривая не гладкая.

\section{Сформулируйте теорему о неявной функции. Допустим кривая $X \subseteq \mathbb{R}^2$ задана уравнением $f(x,y)=0$, и известно, что $\mathrm{grad}f(x_0,y_0)=(2;0)$. Какую из координат $x,y$ можно использовать в качестве локальной координаты на $X$ в окрестности точки $(x_0,y_0)$?}
Пусть есть функция $F: \mathbb{R}^2\xrightarrow{} \mathbb{R}$, для которой выполнены условия:
\begin{enumerate}
\item $F$ определена и непрерывна в окрестности $(x_0,y_0)$ \item $F'_y(x_0,y_0)\not=0$ и $F'_y$ непрерывна в $(x_0,y_0)$ \item $F(x_0, y_0)=0$.
\end{enumerate}
Тогда найдётся окрестность $U_{\delta,\epsilon}(x_0,y_0)=\left\{(x,y)\left|\begin{array}{l}
     x \in (x_0-\delta, x_0 +\delta) \\
     y \in (y_0-\epsilon,y_0+\epsilon)
\end{array}\right.\right\}$ и непрерывная функция $f$ такая, что в $U_{\delta,\epsilon}(x_0,y_0)\quad$ $F(x,y)=0 \Leftrightarrow y = f(x)$ (то есть можно выразить $y$ от $x$ в данной окрестности при выполненных выше условиях).\\
Если кроме всех условий выше $F$ дифференцируема в $U_{\delta,\epsilon}(x_0,y_0)$, то $f$ дифференцируема в $U_{\delta}(x_0)$ и
\begin{gather*}
    f'(x_0)=-\frac{F'_x(x_0,y_0)}{F'_y(x_0,y_0)}
\end{gather*}\\
Задача: проверяем условия теоремы, производная по $x$ не равна нулю, а производная по $y$ равна. Значит в качестве координаты можно взять $y$, а $x$ --- нельзя. Обратите внимание, координата --- эта не та переменная, по которой дифференцируем.

\section{Сформулируйте общую теорему о неявном отображении. Допустим, кривая $X \subseteq \mathbb{R}^3$ задана уравнениями $f(x,y,z)=0,\; g(x,y,z)=0$, и известно, что $\mathrm{grad}f(x_0,y_0,z_0)=(2;0;0)$, $\mathrm{grad}f(x_0, y_0, z_0)=(0;1;3)$. Какие из координат $x,y,z$ можно использовать в качестве локальных координат на $X$ в окрестности точки $(x_0,y_0,z_0)$?}
Обозначения:
$x=(x_1, \dotsc, x_n)$, $y=(y_1, \dotsc y_m)$, $(x, y) = (x_1, \dotsc, x_n, y_1, \dotsc y_m)$.\\
Ещё обозначения: если функции $g_1,\dotsc,g_s$ зависят от $t_1,\dotsc,t_r$, то
\begin{gather*}
    \frac{D(g_1,\dotsc,g_s)}{D(t_1,\dotsc,t_r)} =
    \begin{pmatrix}
        \frac{\partial g_1}{\partial t_1} & \frac{\partial g_1}{\partial t_2} & \cdots & \frac{\partial g_1}{\partial t_r} \\
        \vdots & \vdots & \ddots & \vdots \\
        \frac{\partial g_s}{\partial t_1} & \frac{\partial g_s}{\partial t_2} & \cdots & \frac{\partial g_s}{\partial t_r}
    \end{pmatrix}
\end{gather*}
(по строкам матрицы записаны градиенты (да, в 14 билете градиенты были записаны по столбцам, но так Айз давал на той лекции)).\\\\
Разрешим теперь $m$ уравнений относительно $m$ неизвестных.\\
Теорема: пусть
\begin{enumerate}
    \item $F_1 (x,y),\dotsc,F_m(x,y)$ --- непрерывно дифференцируемы в окрестности точки $(x^{(0)}, y^{(0)})$ (здесь верхние индексы, чтобы не путать с координатами)
    \item $F_j(x^{(0)}, y^{(0)})=0 \quad \forall j=1,\dotsc,m$
    \item det$\frac{D(F_1,\dotsc,F_m)}{D(y_1,\dotsc,y_m)}|_{(x^{(0)}, y^{(0)})}\not=0$
\end{enumerate}
Тогда существует окрестность $U_\delta(x^{(0)}) \times U_\epsilon(y^{(0)})$ и набор дифференцируемых функций $f_1(x_1,\dotsc,x_n),\dots,f_m(x_1,\dotsc,x_m)$, таких что в этой окрестности
\begin{gather*}
    \{F_j(x,y)=0\}^m_{j=1} \Leftrightarrow \{y_j=f_j(x)\}^m_{j=1}
\end{gather*}
при этом $f_j(x^{(0)})=y_j^{(0)}$.\\
Более того,
\begin{gather*}
    \frac{D(f_1,\dotsc, f_m)}{D(x_1,\dotsc,x_n)}\Big|_{x^{(0)}}=-\Big(\frac{D(F_1,\dotsc,F_m)}{D(y_1,\dotsc,y_m)}\Big)^{-1}\Big|_{(x^{(0)}, y^{(0)})} \cdot \frac{D(F_1,\dotsc,F_m)}{D(x_1,\dotsc,x_n)}\Big|_{x^{(0)}}
\end{gather*}\\
Задача: Запишем матрицу
\begin{gather*}
\begin{pmatrix}
    2 & 0 & 0\\
    0 & 1 & 3
\end{pmatrix}
\end{gather*}
Видим, что линейно независимы первый и второй столбец, и первый и третьей. Значит координатой может быть $z$ или $y$. Обратите внимание, если матрица производных по $x$ и $y$ невырождена, то подходит как координата $z$.


\section{Дайте определение касательного вектора к подмножеству $X \subseteq \mathbb{R}^n$ в точке $A \in X$. Как устроено множество всех касательных векторов к гладкому подмногообразию в фиксированной точке?}
Пусть $x^{(0)} \in X \subseteq \mathbb{R}^n$. Построим какую-нибудь кривую, которая целиком лежит в $X$ и проходит через $x^{(0)}$. Пусть эта кривая задаётся параметрически $x_i = \psi_i(s), s \in (-\epsilon, \epsilon)$, $(\psi_1(s), \dotsc, \psi_n(s)) \in X \; \forall s\in (-\epsilon,\epsilon)$ и $(\psi_1(0),\dotsc,\psi_n(0)) = x^{(0)}$. Тогда вектор $(\frac{d\psi_1}{ds}(0),\dotsc,\frac{d\psi_n}{ds}(0))$ называется касательным к $X$ в точке $x^{(0)}$ (если такой вектор определён, конечно).\\
Обратите внимание, касательных векторов может быть бесконечно много, т. к. бесконечно много таких кривых.\\\\\
Пусть $X$ теперь --- гладкое $k$-мерное многообразие и $x_i=\phi_i(t_1,\dotsc,t_k)$ --- гладкие координаты в окрестности точки $x^{(0)} = \Phi(t^{(0)})$. Тогда множество касательных векторов в точке $x^{(0)}$ образуют $k$-мерное векторное пространство (обозначается $T_{x^{(0)}}X$), линейно порождённое следующими векторами
\begin{gather*}
    \Big(\frac{\partial \phi_1}{\partial t_1}(t^{(0)}),\dotsc,\frac{\partial \phi_n}{\partial t_1}(t^{(0)})\Big)\\
    \vdots\\
    \Big(\frac{\partial \phi_1}{\partial t_k}(t^{(0)}),\dotsc,\frac{\partial \phi_n}{\partial t_k}(t^{(0)})\Big)
\end{gather*}
Замечание: эти вектора задают аффинное пространство, чтобы получить геометрическое касательное пространство, нужно сдвинуть $T_{x^{(0)}}X$ в точку $X^{(0)}$.

\section{Допустим, что все точки множества $X \subset \mathbb{R}^n$ удовлетворяют уравнению $f(x)=0$. Докажите, что в любой точке $x^{(0)}\in X$ любой касательный вектор к $X$ перпендикулярен градиенту $\mathrm{grad}f(x^{(0)})$. Опишите касательное пространство к $k$-мерному подмногообразию $\mathbb{R}^n$, заданному системой неявных уравнений (без доказательства).}
Имеем $\forall x \in X \; f(x) = 0$. Тогда для любой кривой $\{x_i=\phi_i(s)\} \subset X$ имеем $f(\phi_1(s),\dotsc,\phi_n(s))=0$. продифференцируем это по $s$, получаем
\begin{gather*}
    \frac{\partial f}{\partial x_1} \cdot \frac{f\phi_1}{ds}+\dotsc + \frac{\partial f}{\partial x_n} \cdot \frac{f\phi_n}{ds} = 0\\
    <\mathrm{grad}f(x^{(0)}),\quad \text{касательный вектор к $X$ в точке $x^{(0)}$}>=0
\end{gather*}
Из того, что скалярное произведении равно нулю, следует, что градиент $f$ перпендикулярен касательному вектору к множеству $X$.\\\\
По предыдущей теореме имеем, что касательное пространство --- ортогональное дополнение к линейной комбинации градиентов неявных уравнений.
\section{Необходимое и достаточное условия локального экстремума для функции нескольких переменных (без доказательства).}
Пусть дана дважды дифференцируемая функция $f(x_1,\dotsc,x_n)$. Необходимое условие: если $f(x^{(0)})$ - локальный экстремум, то $\frac{\partial f}{\partial x_i}(x^{(0)}) = 0 \forall x \in [1;n]$. Точка, в которой выполняется это условие, называется стационарной.\\
Теперь, пусть $x^{(0)}$ --- стационарная точка, и пусть
\begin{gather*}
    D(x^{(0)}) = \begin{vmatrix}
    \frac{\partial^2 f}{\partial^2 x_1} & \frac{\partial^2 f}{\partial x_1 \partial x_2} & \cdots & \frac{\partial^2 f}{\partial x_1 \partial x_n}\\
    \vdots\\
    \frac{\partial^2 f}{\partial x_n \partial x_1} & \frac{\partial^2 f}{\partial x_n \partial x_2} & \cdots & \frac{\partial^2 f}{\partial^2 x_n}
    \end{vmatrix}_{x^{(0)}}
\end{gather*}
Тогда:
\begin{itemize}
    \item Если $D>0$, то $x^{(0)}$ --- локальный экстремум, при этом если $\frac{\partial^2 f}{\partial^2 x_1}>0$, то это локальный минимум, а если $\frac{\partial^2 f}{\partial^2 x_1}<0$, то локальный максимум.
    \item Если $D=0$, то $x^{(0)}$  может как являться локальным экстремумом, так и не являться.
    \item Если $D<0$, то $x^{(0)}$ не является локальным экстремумом.
\end{itemize}
(для справки, эта матрица называется матрицой Гессе, а её определитель --- Гессианом.


\section{Дайте определение точки условного минимума}

Точка $x^{(0)}$ называется условным локальным минимумом подмножества $X \subset \mathbb{R}^n$, если существует окрестность $U(x^{(0)})$, такая что $\forall x \in U(x^{(0)}) \cap X \quad f(x) > f(x^{(0)})$\\
Далее будем считать, что такое множество задаётся набором уравнений вида $\phi(x)=0$.

\section{Сформулируйте теорему о множителях Лагранжа. Объясните идею доказательства в случае, если подмножество $X \subset \mathbb{R}^n$ является гладким многообразием.}

Пусть у нас есть задача вида
\begin{gather*}
    \begin{cases}
        f(x) \xrightarrow{} \mathrm{extr}\\
        \phi_1(x)=0\\
        \vdots\\
        \phi_m(x)=0
    \end{cases}\\
    x \in \mathbb{R}^n\\
    m < n
\end{gather*}
Функцией Лагранжа называется
\begin{gather*}
    L(x,\lambda) = f(x) - \sum_{i=1}^m \lambda_i g_i(x)\\
    x \in \mathbb{R}^n\\
    \lambda \in \mathbb{R}^m
\end{gather*}
$\lambda$ называют множителями Лагранжа.\\
Теорема: пусть $x^{(0)}$ --- точка условного локального экстремума в задаче выше, и пусть в окрестности точки $x^{(0)}$ $X$ --- гладкое многообразие. Тогда существуют такие $\lambda^{(0)}$, что точка $(x^{(0)},\lambda^{(0)}) = (x^{(0)}_1,\dotsc, x^{(0)}_n, \lambda^{(0)}_1, \dotsc, \lambda^{(0)}_m) \in \mathbb{R}^{m+n}$ является стационарной для $L(x, \lambda)$.\\
То есть
\begin{gather*}
\frac{\partial L}{\partial x_i}(x^{(0)},\lambda^{(0)}) = 0 \quad\forall i \in [1;n]\\
\frac{\partial L}{\partial \lambda_i}(x^{(0)},\lambda^{(0)}) = 0 \quad\forall i \in [1;m]\\
\end{gather*}
Второе в силу линейности по $\lambda$ эквивалентно $g_i(x^{(0)})=0$, что означает, что $x^{(0)} \in X$.\\
Посмотрим теперь на первое
\begin{gather*}
    \frac{\partial L}{\partial x_j} = \frac{\partial}{\partial x_j} (f(x) - \sum_{i=1}^m \lambda_i g_i(x)) = \frac{\partial  f}{\partial x_j} - \sum_{i=1}^m \lambda_i \frac{\partial g_i}{\partial x_j} = 0\\
    \begin{pmatrix}
        \frac{\partial f}{\partial x_1} \\
        \vdots\\
        \frac{\partial f}{\partial x_n}
    \end{pmatrix}
    - \lambda_1 \begin{pmatrix}
        \frac{\partial g_1}{\partial x_1} \\
        \vdots\\
        \frac{\partial g_1}{\partial x_n}
    \end{pmatrix}
    - \dotsc
    - \lambda_m \begin{pmatrix}
        \frac{\partial g_m}{\partial x_1} \\
        \vdots\\
        \frac{\partial g_m}{\partial x_n}
    \end{pmatrix} = \begin{pmatrix}
        0\\
        \vdots\\
        0
    \end{pmatrix}
\end{gather*}
Это значит, что
\begin{gather*}
    \mathrm{grad}f = \sum_{i=1}^m \lambda_i \mathrm{grad}g_i
\end{gather*}
Так как, все наши переходы были равносильными, нам осталось доказать, что найдутся такие $\lambda$, то есть, что $\mathrm{grad}f$ является линейной комбинацией $\mathrm{grad}g_i$ в данной точке.\\
Поскольку $X$ гладкая в точке $x^{(0)}$, будем предполагать, что $X$ удовлетворяет условию следствия из теоремы о неявном отображении, то есть градиенты $\mathrm{grad}g_i$ линейно независимы. Без ограничения общности будем считать $x^{(0)}\in X \subset \mathbb{R}^n$ --- точка условного локального минимума. Тогда, если возьмём какую-нибудь кривую $\{x_i=\phi(t)\} \subseteq X$, такую, что $\phi_i(0)=x_i^{(0)}$, то на ней это также будет точка локального минимума, запишем касательный вектор
\begin{gather*}
    u = (\frac{d\phi_1}{dt}(0),\dotsc,\frac{d\phi_n}{dt}(0)) \in T_{x^{(0)}}X
\end{gather*}
Функция $\alpha(t)=f(\phi_1(t),\dotsc,\phi_n(t))$ имеет в $t=0$ локальный минимум. По теореме Ферма $\frac{d \alpha}{dt}(0) = 0$. А это
\begin{gather*}
    \frac{\partial f}{\partial x_1}(x^{(0)})\cdot \frac{d \phi_1}{dt}(0)+
    \dotsc + \frac{\partial f}{\partial x_n}(x^{(0)})\cdot \frac{d \phi_n}{dt}(0) = <\mathrm{grad}f(x^{0}),u> = 0
\end{gather*}
Таким образом, градиент целевой функции в точки экстремума перпендикулярен любому касательному вектору $u \in T_{x^{(0)}}X$, то есть $\mathrm{grad}f(x^{(0)}) \perp T_{x^{(0)}}X$. А это значит, что этот градиент лежит в ортогональном дополнении
\begin{gather*}
    \mathrm{grad}f(x^{(0)}) \in (T_{x^{(0)}}X)^\perp = <\mathrm{grad}g_1(x^{(0)}),\dotsc,\mathrm{grad}g_m(x^{(0)})>
\end{gather*}
А раз $\mathrm{grad}f(x^{(0)})$ лежит в линейной оболочке$\mathrm{grad}g_i(x^{(0)})$, то он является их линейной комбинацией.
	


\end{document}