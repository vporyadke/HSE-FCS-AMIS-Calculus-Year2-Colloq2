\ProvidesFile{Q22.tex}[Билет 22]

\section{Сформулируйте достаточное условие в методе множителей Лагранжа. Объясните, как на практике
проверять это условие (скажем, с помощью примера).}
\begin{remark}
    Пусть $Q = Q(y_1, ..., y_n) = \sum\limits_{i, j = 1}^n a_{ij} y_i y_j$ -- квадратичная форма
    на $\mathbb{R}^n = V$. Пусть $W \subset V$ -- векторное подпространство. Тогда определена операция ограничения $Q$
    на подпространство $W$: $Q|_W$
\end{remark}

\begin{theorem}{\textbf{Достаточное условие локального условного экстремума.}}\\
    Пусть $X = \left\{g_i(x) = 0 | \; i = 1,..., m\right\} \subset \mathbb{R}^n$ в точке $x^{(0)} \in X$ является гладким многообразием и
    $f: \mathbb{R}^n \to \mathbb{R}$ -- целевая функция (та, которую хотим оптимизировать на этом многообразии).
    Ограничим квадратичную форму $d^2 L$ на касательное пространство к $X$ в точке $x^{(0)}$:
    $$d^2 L(x^{(0)}, \lambda^{(0)})|_{T_{x^{(0)}}X}$$
    Допустим, в точке $x^{(0)}$ выполнены условия Лагранжа:
    $$gradf|_{x^{(0)}} = \sum\limits_{i = 1}^m \lambda^{(0)}_i gradg_i|_{x^{(0)}} \text{ и } g_{i}(x^{(0)}) = 0$$
    Тогда:
    \begin{itemize}
        \item Если квадратичная форма положительно определена при этом ограничении, то $x^{(0)}$ -- точка условного локального минимума для $f$;
        \item Если квадратичная форма отрицательно определена при этом ограничении, то $x^{(0)}$ -- точка условного локального максимума для $f$;
        \item Если квадратичная форма знаконеопределённая при этом ограничении, то $x^{(0)}$ не является точкой условного локального экстремума для $f$;
        \item Если квадратичная форма неотрицательно или неположительно определена при этом ограничении, то теорема ничего не говорит о точке $x^{(0)}$.
    \end{itemize}
\end{theorem}

\textbf{Пример 1.}
Оптимизируем $f = 3x + 4y$ на окружности $x^2 + y^2 - 1 = 0$. Запишем функцию Лагранжа:
\[
    L(x, \, y, \, \lambda) = 3x + 4y - \lambda(x^2 + y^2 - 1)
\]
Найдём стационарные точки:
\[
    \systeme*{L^{'}_{x} = 0, L^{'}_{y} = 0, L^{'}_{\lambda} = 0} \iff
    \systeme*{3 - 2\lambda x = 0, 4 - 2\lambda y = 0, x^2 + y^2 - 1 = 0} \iff
    \systeme*{x = \frac{3}{2\lambda}, y = \frac{4}{2\lambda}, \left(\frac{3}{2\lambda}\right)^{2} + \left(\frac{4}{2\lambda}\right)^{2} - 1 = 0} \iff
    \systeme*{\lambda_1 = \frac{5}{2}, x_1 = \frac{3}{5}, y_1 = \frac{4}{5}} \text{или}
    \systeme*{\lambda_2 = -\frac{5}{2}, x_2 = -\frac{3}{5}, y_2 = -\frac{4}{5}}
\]
Исследуем точку $(x_1, \, y_1)$ при помощи условия второго порядка:
\[
    d^2 L = L^{''}_{xx}dx^2 + 2L^{''}_{xy}dxdy + L^{''}_{yy}dy^2 + 2L^{''}_{x\lambda}dxd\lambda + 2L^{''}_{y\lambda}dyd\lambda + L^{''}_{\lambda\lambda}d\lambda^2
\]
Заметим, что $L^{''}_{\lambda\lambda}d\lambda^2 = 0$, т.к. $L$ линейна по $\lambda$ (это верно всегда, а не только в этом примере).
Теперь найдём полный дифференциал условия:
\[
    x^2 + y^2 - 1 = 0 \Rightarrow d(x^2 + y^2 - 1) = d0 \Rightarrow 2xdx + 2ydy = 0\\
\]
Также заметим, что $L^{''}_{x\lambda} = -g^{'}_{x}$, $L^{''}_{y\lambda} = -g^{'}_{y}$. Тогда:
\[
    L^{''}_{x\lambda}dxd\lambda + L^{''}_{y\lambda}dyd\lambda = d\lambda(g^{'}_{x}dx + g^{'}_{y}dy) = d\lambda dg = 0 \text{ (это верно всегда)}
\]
Вычислим $d^2 L$ в точке $(\frac{3}{5}, \, \frac{4}{5}, \, \frac{5}{2})$:
\[
    d^2 L(\frac{3}{5}, \, \frac{4}{5}, \, \frac{5}{2}) = -5dx^2 - 5dy^2
\]
Видим, что, без всякого ограничения, квадратичная форма $d^2 L$ в нужной точке отрицательно определена $\Rightarrow$
при ограничении форма $d^2 L|_{T_{(x_1, \, y_1)} X}$ тоже определена отрицательно $\Rightarrow (x_1, \, y_1)$ -- точка локального максимума.
Вторая точка исследуется аналогично.

\textbf{Пример 2.}
Оптимизируем $f = 2x^2 - y^2$ при условии $e^y - \sin x - 1 = 0$. Запишем функцию Лагранжа:
\[
    L(x, \, y, \, \lambda) = 2x^{2} - y^{2} - \lambda(e^{y} - \sin x - 1)
\]
Найдём стационарные точки (аналогично примеру 1):
\[\left\{\begin{aligned}
    4x + \lambda \cos x &= 0 \\
    -2y - \lambda e^y &= 0 \\
    e^y - \sin x - 1 &= 0
\end{aligned}\right.\]
Исследуем решение $(x_0, \, y_0, \lambda_0) = (0, \, 0, \, 0)$ при помощи условия второго порядка:
\[
    d^2 L = (4 - \lambda \sin x) dx^2 + (-2 - \lambda e^y)dy^2
\]
Подставим точку $(0, \, 0, \, 0)$:
\[
    d^2 L = 4dx^2 - 2dy^2 \text{ -- знаконеопределённая квадратичная форма}
\]
Необходимо исследовать ограничение $d^2 L|_{T_{x_0, y_0}X}$. Возьмём полный дифференциал от уравнения связи:
\[
    dg = d(e^y - \sin x - 1) = e^ydy - \cos xdx = 0
\]
Подставим точку $(0, \, 0)$:
\[
    dg(0, 0) = dy - dx = 0 \Rightarrow dy = dx
\]
Таким образом, <<ограничить $d^2 L$ на $T_{x^{(0)}}X$ означает <<подставить в $d^2 L$ линейные соотношения на дифференциалы,
полученные из уравнения $dg = 0$>>. Подставим в  $4dx^2 - 2dy^2$ уравнение $dy = dx$:
\[
    d^2 L|_{T_{(x_0, \, y_0)}X} = 2dx^2 > 0 \Rightarrow (0, 0) \text{ -- точка условного локального минимума.}
\]